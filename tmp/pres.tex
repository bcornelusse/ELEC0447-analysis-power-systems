\begin{frame}{Energy planning and EROI}
\phantomsection\label{energy-planning-and-eroi}
\end{frame}

\begin{frame}{Bertrand Cornélusse}
\phantomsection\label{bertrand-cornuxe9lusse}
2022-10-14 ENVT3065
\end{frame}

\begin{frame}{Plan}
\phantomsection\label{plan}
Introduction

World energy outlook

Energy planning

Planning to maximize «~Energy return on energy invested~», the case of
Belgium
\end{frame}

\begin{frame}{Introduction}
\phantomsection\label{introduction}
\end{frame}

\begin{frame}{Power and energy}
\phantomsection\label{power-and-energy}
\end{frame}

\begin{frame}{Introduction: an e-bike}
\phantomsection\label{introduction-an-e-bike}
\begin{itemize}
\tightlist
\item
  Hints: the size of the battery, and the power of the motor.
\item
  What does the battery store?

  \begin{itemize}
  \tightlist
  \item
    Some energy measured in joules (J) or watt-hours (Wh): 1Wh = 3600J.
  \end{itemize}
\item
  How fast is the energy delivered?

  \begin{itemize}
  \tightlist
  \item
    It depends on the power of the motor, measured in watts (W)
  \end{itemize}
\item
  Good to know: an e-bike has a typical battery size of 500Wh, and the
  power of the motor is 250W.
\end{itemize}

You want to buy an electric bike. You already

decided on the type, the color, etc. How do you

decide on the electrical part? What are the

typical options you could have?
\end{frame}

\begin{frame}{Power definition}
\phantomsection\label{power-definition}
\includegraphics{img/Energy planning and EROI_1.png}

\includegraphics{img/Energy planning and EROI_2.png}
\end{frame}

\begin{frame}{Units of power and energy in energy systems}
\phantomsection\label{units-of-power-and-energy-in-energy-systems}
\begin{itemize}
\tightlist
\item
  Power typically ranges from

  \begin{itemize}
  \tightlist
  \item
    kW e.g. PV panels
  \item
    to GW e.g. nuclear power plant
  \end{itemize}
\item
  Energy typically ranges from

  \begin{itemize}
  \tightlist
  \item
    Several kWh: daily production of a residential PV plant, or 1l of
    oil
  \item
    To thousands of TWh: consumption of a country over a year
  \item
    One ExaJoule (EJ) = 277,78 TWh
  \item
    ``The ton of oil equivalent (toe) represents the quantity of energy
    contained in a ton of crude oil, that is gigajoules 41.868. This
    unit is used to express and compare energies of different sources.
    According to the international conventions, one ton of oil
    equivalent amounts for example in 1 616 kg of coal, 1 069 m3 of gas
    from Algeria or 954 kg of gasoline. For electricity, 1 toe is worth
    11.6 MWh {(caution here)} .'' (from
    \url{https://www.insee.fr/en/metadonnees/definition/c1355})
  \end{itemize}
\end{itemize}

\includegraphics{img/Energy planning and EROI_3.png}
\end{frame}

\begin{frame}{Robert and the toaster: https://youtu.be/S4O5voOCqAQ}
\phantomsection\label{robert-and-the-toaster-httpsyoutu.bes4o5voocqaq}
\end{frame}

\begin{frame}{Renewable energy sources}
\phantomsection\label{renewable-energy-sources}
\end{frame}

\begin{frame}{Definition}
\phantomsection\label{definition}
\begin{itemize}
\tightlist
\item
  Renewable energy
  is~\href{https://unece.org/DAM/energy/se/pdfs/comm25/ECE_ENERGY_2016_4.pdf}{energy
  derived from natural sources~}that are replenished at a higher rate
  than they are consumed.

  \begin{itemize}
  \tightlist
  \item
    Sunlight and wind, for example, are such sources that are constantly
    being replenished. Renewable energy sources are plentiful and all
    around us.
  \end{itemize}
\item
  Fossil fuels - coal, oil, and gas - on the other hand, are
  non-renewable resources that take hundreds of millions of years to
  form. Fossil fuels, when burned to produce energy, cause harmful
  greenhouse gas emissions, such as carbon dioxide.
\end{itemize}

Source: https://www.un.org/en/climatechange/what-is-renewable-energy
\end{frame}

\begin{frame}{Solar energy}
\phantomsection\label{solar-energy}
Solar energy is the most abundant of all energy resources (\ldots).

The rate at which solar energy is intercepted by the Earth is
about~\href{https://www.ipcc.ch/site/assets/uploads/2018/03/Chapter-3-Direct-Solar-Energy-1.pdf}{10,000
times greater}~than the rate at which humankind consumes energy.

Solar technologies can deliver heat, natural lighting, electricity,
\ldots{}

Electrical energy either through photovoltaic panels or through mirrors
that concentrate solar radiation.

(\ldots) A significant contribution to the energy mix from direct solar
energy is possible for every country.

The cost of manufacturing solar panels has plummeted dramatically in the
last decade, making them affordable and often the cheapest form of
electricity. Solar panels have~\href{https://www.irena.org/solar}{a
lifespan of roughly 30 years}, \ldots.

Source: https://www.un.org/en/climatechange/what-is-renewable-energy
\end{frame}

\begin{frame}{Photovoltaic generation}
\phantomsection\label{photovoltaic-generation}
\begin{itemize}
\tightlist
\item
  A PV cell is composed of semiconductor material. Photons emitted by
  the sun interact with the semiconducting material in two ways:

  \begin{itemize}
  \tightlist
  \item
    photons directly transmit energy to electrons and allow them to move
    into the conduction band.
  \item
    a thermally generated current as in a p-n junction (diode).
  \end{itemize}
\item
  See \url{https://youtu.be/L_q6LRgKpTw}
\end{itemize}
\end{frame}

\begin{frame}{Effect of irradiance and temperature}
\phantomsection\label{effect-of-irradiance-and-temperature}
\includegraphics{img/Energy planning and EROI_4.png}

{Sources:}

{\url{https://en.wikipedia.org/wiki/Theory_of_solar_cells}}
{\url{https://en.wikipedia.org/wiki/Maximum_power_point_tracking}}

\includegraphics{img/Energy planning and EROI_5.png}
\end{frame}

\begin{frame}{Photovoltaic power production potential}
\phantomsection\label{photovoltaic-power-production-potential}
Source: \url{https://solargis.com/maps-and-gis-data/download/europe}

How many kWh per kWP in Belgium?

Note: also depends on the orientation of PV panels and other factors
\end{frame}

\begin{frame}{PV variability}
\phantomsection\label{pv-variability}
\includegraphics{img/Energy planning and EROI_6.png}

\includegraphics{img/Energy planning and EROI_7.png}
\end{frame}

\begin{frame}{Wind energy}
\phantomsection\label{wind-energy}
Wind energy harnesses the kinetic energy of moving air by using large
wind turbines located on land (onshore) or in sea- or freshwater
(offshore). Wind energy has been used for millennia, but onshore and
offshore wind energy technologies have evolved over the last few years
to maximize the electricity produced - with taller turbines and larger
rotor diameters.

Though average wind speeds vary considerably by location, the
world's~\href{https://www.ipcc.ch/site/assets/uploads/2018/03/Chapter-7-Wind-Energy-1.pdf}{technical
potential for wind energy}~exceeds global electricity production, and
ample potential exists in most regions of the world to enable
significant wind energy deployment.

Many parts of the world have strong wind speeds, but the best locations
for generating wind power are sometimes remote ones. Offshore wind power
offers t\href{https://www.irena.org/wind}{remendous potential}.
\end{frame}

\begin{frame}{Wind turbines}
\phantomsection\label{wind-turbines}
\includegraphics{img/Energy planning and EROI_8.jpg}

\includegraphics{img/Energy planning and EROI_9.png}
\end{frame}

\begin{frame}{Power conversion}
\phantomsection\label{power-conversion}
\includegraphics{img/Energy planning and EROI_10.png}
\end{frame}

\begin{frame}{Turbine efficiency}
\phantomsection\label{turbine-efficiency}
\includegraphics{img/Energy planning and EROI_11.png}

\includegraphics{img/Energy planning and EROI_12.png}
\end{frame}

\begin{frame}{Wind generator operating characteristic}
\phantomsection\label{wind-generator-operating-characteristic}
\includegraphics{img/Energy planning and EROI_13.png}

\includegraphics{img/Energy planning and EROI_14.png}
\end{frame}

\begin{frame}{Wind atlas}
\phantomsection\label{wind-atlas}
Source: \url{https://globalwindatlas.info/en}

How many MWh per MWP in Belgium?

Offshore vs. onshore?
\end{frame}

\begin{frame}{Other sources}
\phantomsection\label{other-sources}
Geothermal energy

Hydropower

Bioenergy

\includegraphics{img/Energy planning and EROI_15.png}
\end{frame}

\begin{frame}{CHG emissions related to electricity consumption}
\phantomsection\label{chg-emissions-related-to-electricity-consumption}
\url{https://app.electricitymaps.com/map}
\end{frame}

\begin{frame}{World energy outlook}
\phantomsection\label{world-energy-outlook}
\end{frame}

\begin{frame}{World energy consumption}
\phantomsection\label{world-energy-consumption}
\includegraphics{img/Energy planning and EROI_16.png}
\end{frame}

\begin{frame}{Consumption-based CO2 emissions per capita vs GDP per
capita, 2019}
\phantomsection\label{consumption-based-co2-emissions-per-capita-vs-gdp-per-capita-2019}
\includegraphics{img/Energy planning and EROI_17.png}

CO2 emissions are directly related to our consumption / production

\url{https://ourworldindata.org/worlds-energy-problem}
\end{frame}

\begin{frame}{Share of primary energy from renewable sources, 2021}
\phantomsection\label{share-of-primary-energy-from-renewable-sources-2021}
\includegraphics{img/Energy planning and EROI_18.png}
\end{frame}

\begin{frame}{Annual change in renewable energy generation, 2021}
\phantomsection\label{annual-change-in-renewable-energy-generation-2021}
\includegraphics{img/Energy planning and EROI_19.png}

Source:
\url{https://ourworldindata.org/grapher/annual-change-renewables}
\end{frame}

\begin{frame}{See also}
\phantomsection\label{see-also}
A great references to dig: is IEA's, World Energy Outlook. The latest
one: \url{https://www.iea.org/reports/world-energy-outlook-2021}
\end{frame}

\begin{frame}{Energy planning}
\phantomsection\label{energy-planning}
\end{frame}

\begin{frame}{Definition of energy planning}
\phantomsection\label{definition-of-energy-planning}
\begin{itemize}
\tightlist
\item
  Energy planning aims at determining in which types of energy sources,
  transmission and conversion systems we want to invest to satisfy some
  needs for

  \begin{itemize}
  \tightlist
  \item
    Mobility
  \item
    Industry
  \item
    Heating
  \item
    Agriculture
  \item
    Etc.
  \end{itemize}
\item
  The plan should be sustainable and in line with CO2 emission targets
\end{itemize}
\end{frame}

\begin{frame}{Energy sources}
\phantomsection\label{energy-sources}
Renewable energy sources

Nuclear energy

Fossil fuels
\end{frame}

\begin{frame}{Transmission and distribution systems - electricity}
\phantomsection\label{transmission-and-distribution-systems---electricity}
\includegraphics{img/Energy planning and EROI_20.png}

\includegraphics{img/Energy planning and EROI_21.png}

\url{https://www.elia.be/fr/donnees-de-reseau/transport/flux-de-bouclage-par-frontiere/flux-physiques-sur-le-reseau-haute-tension-belge}
\end{frame}

\begin{frame}{Transmission and distribution systems - gas}
\phantomsection\label{transmission-and-distribution-systems---gas}
\includegraphics{img/Energy planning and EROI_22.png}

https://www.entsog.eu/sites/default/files/2018-10/ENTSOG\_CAP\_MAY2015\_A0FORMAT.pdf

See also https://www.fluxys.com/en/company/fluxys-belgium/infrastructure

Other networks: oil, wood, etc.
\end{frame}

\begin{frame}{Conversion systems -- clean hydrogen}
\phantomsection\label{conversion-systems-clean-hydrogen}
\begin{itemize}
\tightlist
\item
  Hydrogen is necessary for industry, and can be a solution for long
  term storage of renewable energy (e.g. to cope with seasonal variation
  of PV generation)
\item
  Hydrogen currently mostly generated from fossiel fuels

  \begin{itemize}
  \tightlist
  \item
    Need to move towards clean hydrogen. What is it?
  \end{itemize}
\end{itemize}

\includegraphics{img/Energy planning and EROI_23.png}

\includegraphics{img/Energy planning and EROI_24.jpg}
\end{frame}

\begin{frame}{What about the efficiency of clean hydrogen production?}
\phantomsection\label{what-about-the-efficiency-of-clean-hydrogen-production}
Considering the industrial production of hydrogen, and using current
best processes for water electrolysis (PEM or alkaline electrolysis)

which have an effective electrical efficiency of
70--80\%,\href{https://en.m.wikipedia.org/wiki/Electrolysis_of_water\#cite_note-46}{{[}46{]}}\href{https://en.m.wikipedia.org/wiki/Electrolysis_of_water\#cite_note-47}{{[}47{]}}\href{https://en.m.wikipedia.org/wiki/Electrolysis_of_water\#cite_note-48}{{[}48{]}}

producing 1~kg of hydrogen (which has
a~\href{https://en.m.wikipedia.org/wiki/Specific_energy}{specific
energy}~of 143 MJ/kg) requires 50--55~kWh (180--200~MJ) of electricity.

As of 2022, different analysts predict annual manufacture of equipment
by 2030 as 47 GW, 104 GW and 180 GW,
respectively.\href{https://en.m.wikipedia.org/wiki/Electrolysis_of_water\#cite_note-50}{{[}50{]}}

Source: https://en.m.wikipedia.org/wiki/Electrolysis\_of\_water
\end{frame}

\begin{frame}{Note on clean hydrogen}
\phantomsection\label{note-on-clean-hydrogen}
\begin{itemize}
\tightlist
\item
  \emph{Green hydrogen has been hailed as a clean energy source for the
  future. }
\item
  \emph{But the gas itself is invisible -- so why are so many colourful
  descriptions used when referring to it?}

  \begin{itemize}
  \tightlist
  \item
    \emph{It all comes down to the way it is produced. Hydrogen emits
    only water when burned. But creating it can be carbon intensive.}
  \item
    \emph{So various ways to lessen this impact have been developed --
    and scientists assign colours to the different types to distinguish
    between them.}
  \end{itemize}
\item
  \emph{Depending on production methods, hydrogen can be grey, blue or
  green -- and sometimes even pink, yellow or turquoise -- although
  naming conventions can vary across countries and over time.}
\item
  { \emph{But green hydrogen is the only type produced in a
  climate-neutral manner, meaning it could play a~} } {
  \emph{\href{https://www.iea.org/reports/the-future-of-hydrogen}{vital
  role}} } { \emph{~in global efforts to} } {
  \emph{\href{https://www.un.org/en/climatechange/net-zero-coalition}{~reduce
  emissions to net zero by 2050}} } { \emph{.} }
\end{itemize}

Source:
https://www.weforum.org/agenda/2021/07/clean-energy-green-hydrogen/

\includegraphics{img/Energy planning and EROI_25.png}
\end{frame}

\begin{frame}{Conversion systems --Synthetic fuels}
\phantomsection\label{conversion-systems-synthetic-fuels}
\end{frame}

\begin{frame}{Energy planning is a very complex problem}
\phantomsection\label{energy-planning-is-a-very-complex-problem}
\begin{itemize}
\tightlist
\item
  Energy impacts all aspects of life
\item
  There is much uncertainty on technologies, efficiencies, availability
  of resources
\item
  Networks must be built to transmit energy in various forms
\item
  We must work on the demand side as well
\item
  We inherit from the current situation
\item
  State budgets are limited
\item
  Transition should be fast
\item
  Many actors make decisions =\textgreater{} huge need for coordination
\item
  \ldots{} There is no easy solution
\end{itemize}
\end{frame}

\begin{frame}{The energy scope model}
\phantomsection\label{the-energy-scope-model}
{ \emph{The EnergyScope project is an open-source whole-energy system
for regional energy system. The model optimises the design and hourly
operation over a year for a target year.} }

{ \emph{EnergyScope is mainly developped by EPFL (Switzerland) and
UCLouvain (Belgium). See~} } {
\emph{\href{https://energyscope-td.readthedocs.io/en/master/sections/Releases.html}{Releases}}
} { \emph{~section for acknowledgment, versions and publications.} }

\includegraphics{img/Energy planning and EROI_26.png}

{Reference: Gauthier Limpens, Stefano Moret, Hervé Jeanmart, and
Francois Maréchal. EnergyScope TD: A novel open-source model for
regional energy systems.~} { \emph{Applied Energy} } {,
255(March):113729, 2019.}

Source:
\href{https://energyscope-td.readthedocs.io/en/master/index.html}{https://energyscope-td.readthedocs.io/en/master/index.html\#}
\end{frame}

\begin{frame}{The planning problem is modeled as a large linear
optimisation problem}
\phantomsection\label{the-planning-problem-is-modeled-as-a-large-linear-optimisation-problem}
\includegraphics{img/Energy planning and EROI_27.png}

{~} { \emph{Overview of the LP modeling framework} }

Source:
\url{https://energyscope-td.readthedocs.io/en/master/sections/Model\%20formulation.html}
\end{frame}

\begin{frame}{What is an optimisation problem?}
\phantomsection\label{what-is-an-optimisation-problem}
Mathematical \textbf{programming} is a field of applied mathematics that
deals with solving optimization problems.

It provides a framework and solution methods for computing the decisions
of an optimization problem, given an \emph{objective function to
minimize or maximize} and \emph{constraints} on the \emph{decision
variables} .

Mathematical programming relies on a model of the problem.

There is a large variety of mathematical programming problem types,
depending on the characteristics of the objective function and of the
constraints, and the restrictions that apply to variables.
\end{frame}

\begin{frame}{Linear program, mathematical formulation}
\phantomsection\label{linear-program-mathematical-formulation}
If the objective is linear and the constraints are linear, we talk about
linear programming (LP) or linear optimization.

These problems are nowadays easy to solve, even for large problems
(millions of decision variables), with standard computers.

\includegraphics{img/Energy planning and EROI_28.png}
\end{frame}

\begin{frame}{Graphic representation}
\phantomsection\label{graphic-representation}
\includegraphics{img/Energy planning and EROI_29.png}
\end{frame}

\begin{frame}{Details of the energy scope model}
\phantomsection\label{details-of-the-energy-scope-model}
\begin{itemize}
\tightlist
\item
  satisfy the system end-use demand (EUD) instead of final energy
  consumption

  \begin{itemize}
  \tightlist
  \item
    The system end-use demand comprises electricity, heat, transport,
    and non-energy demands. For instance, passenger mobility is defined
    in passenger kilometers per year rather than in a certain amount of
    gasoline to fuel cars or electricity to power trains;
  \end{itemize}
\item
  optimizing the system design and operation by minimizing its overall
  cost;
\item
  using an hourly resolution which makes the model suitable for
  analyzing the integration of intermittent renewable energy resources
  and storage;
\item
  modeling the country as a single node where transmissions constraints
  within the country are not considered
\item
  For details, see
  \url{https://energyscope-td.readthedocs.io/en/master/sections/Model\%20formulation.html}
\end{itemize}
\end{frame}

\begin{frame}{Energy planning and EROI}
\phantomsection\label{energy-planning-and-eroi-1}
{Summary} { of }

{Dumas, Jonathan, Antoine Dubois, Paolo Thiran, Pierre Jacques,
Francesco } {Contino} {, Bertrand Cornélusse, and Gauthier } {Limpens}
{. ``The } {energy} { return on } {investment} { of } {whole} { }
{energy} { } {systems} {: application to } {Belgium} {.``~} {
\emph{arXiv} } { \_ \_ } { \emph{preprint} } { \_ arXiv:2205.06727\_ }
{~(2022). To } {appear} { in } {
\emph{\href{https://www.springer.com/journal/41247/}{Biophysical}} } {
\emph{\href{https://www.springer.com/journal/41247/}{}} } {
\emph{\href{https://www.springer.com/journal/41247/}{Economics}} } {
\emph{\href{https://www.springer.com/journal/41247/}{and}} } {
\emph{\href{https://www.springer.com/journal/41247/}{Sustainability}} }
{ \emph{.} }
\end{frame}

\begin{frame}{Motivation}
\phantomsection\label{motivation}
Planning the defossilization of energy systems while maintaining access
to abundant primary energy resources is a nontrivial multi-objective
problem encompassing economic, technical, environmental, and social
aspects.

\textbf{However, most long-term policies consider the cost of the system
as the leading indicator in the energy system models to decrease the
carbon footprint.}
\end{frame}

\begin{frame}{What is the EROI: Energy return on energy invested}
\phantomsection\label{what-is-the-eroi-energy-return-on-energy-invested}
\includegraphics{img/Energy planning and EROI_30.png}
\end{frame}

\begin{frame}{Comments}
\phantomsection\label{comments}
\begin{itemize}
\tightlist
\item
  The lower the EROI of an energy source, the more input energy is
  required to produce the output energy, which results in less net
  energy available for consumption.
\item
  The literature about EROI focuses mainly on three main fields of
  research:

  \begin{itemize}
  \tightlist
  \item
    (1) the link between EROI and societal well-being;
  \item
    (2) estimation of the EROI of an energy resource or technology;
  \item
    (3) estimation of the EROI at the level of an economy or a society.
  \end{itemize}
\end{itemize}
\end{frame}

\begin{frame}{(1) Link between EROI and societal well-being}
\phantomsection\label{link-between-eroi-and-societal-well-being}
The estimated societal EROI is correlated with the \textbf{Human
Development Index} (HDI), which is a standard living indicator.

However, for a few countries with a high level of development, HDI above
0.75, there is a saturation point where increasing the EROI above 20 is
not associated with further improvement in society.

In addition, the relationship between EROI and HDI is non-linear as the
HDI increases less and less rapidly with societal EROI.
\end{frame}

\begin{frame}{(2) Estimation of the EROI of an energy resource or
technology}
\phantomsection\label{estimation-of-the-eroi-of-an-energy-resource-or-technology}
\begin{itemize}
\tightlist
\item
  The characteristics of the primary energy sources, including the EROI
  of each fuel, are investigated by Hall et al. {[}7{]}.
\item
  They conclude that:

  \begin{itemize}
  \tightlist
  \item
    (i) the EROI of critical fuels, such as oil and gas, is declining;
  \item
    (ii) most renewable and non-conventional energy alternatives have
    substantially lower EROI values than traditional conventional fossil
    fuels.
  \end{itemize}
\item
  Another more recent study {[}8{]} estimates the EROI of fossil fuels
  at both primary and final energy stages. However, their results
  suggest that the current EROIoffossil fuels may not differ from the
  EROI of renewables, which illustrates the difficulty of adequately
  assessing the EROI of resources or technologies.
\end{itemize}

{References relate to «~Dumas, Jonathan, Antoine Dubois, Paolo Thiran,
Pierre Jacques, Francesco Contino, Bertrand Cornélusse, and Gauthier
Limpens. ``The energy return on investment of whole energy systems:
application to Belgium.''~} { \emph{arXiv preprint arXiv:2205.06727} }
{~(2022). To appear in } {
\emph{\href{https://www.springer.com/journal/41247/}{Biophysical
Economics and Sustainability}} } { \emph{.~»} }

\includegraphics{img/Energy planning and EROI_31.png}

Source: {DUPONT, Elise, JEANMART, Hervé, POSSOZ, Louis,~} { \emph{et
al.} } {~} { \emph{Transition énergétique et (dé) croissance économique}
} {. Université catholique de Louvain, Institut de Recherches
Economiques et Sociales (IRES), 2017.}

\url{https://www.regards-economiques.be/images/reco-pdf/reco_175.pdf}
\end{frame}

\begin{frame}{(3) The estimation of the EROI at the level of an economy
or a society}
\phantomsection\label{the-estimation-of-the-eroi-at-the-level-of-an-economy-or-a-society}
This is precisely the topic of the paper. Why?

It is not relevant to compare the EROI of renewable resources or
technologies independently. For instance, solar and wind energies are
intermittent and stochastic. Gas and nuclear power plants are adjustable
and can meet fluctuating demand. Thus, comparing the EROI of solar vs.
nuclear without considering storage systems and other assets to balance
the system is not pertinent.

Second, a whole energy system comprises several sectors (mobility, heat,
electricity, industry) that use several technologies and resources that
can be imported or extracted. These resources are transported, stored,
and converted by energy conversion technologies to supply end-use
demands such as electricity, transport, heating, and the production of
goods. Assessing an energy system as a whole opens the opportunity for
the full deployment of synergies and generates unexpected results
{[}15{]}. Thus, the EROI of the system cannot be the sum of the EROI of
each of its components.

EnergyScope was adapted to work with the EROI as an objective
function.The case of Belgium in 2035 was investigated.
\end{frame}

\begin{frame}{Results}
\phantomsection\label{results}
\includegraphics{img/Energy planning and EROI_32.png}

\includegraphics{img/Energy planning and EROI_33.png}
\end{frame}
